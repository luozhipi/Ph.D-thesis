\chapter*{Preface}
\pdfbookmark[0]{Preface}{dunno-preface}%
\markboth{\itshape Preface}{}

	This dissertation is all about the fascinating topic of
        syntax. Such a statement will probably be frowned upon by many
        computer scientists and programmers. Syntax, in particular
        textual syntax, seems to be a rather unpopular topic these
        days. Grammars and parsing are not particularly active
        research topics, resulting in a lack of innovation in one of
        the most foundational topics of computer science. In fact,
        textual syntax is so unpopular that people try to avoid it
        altogether and attempt to move to development methods without
        a textual syntax. However, instead of abandoning textual
        syntax for its problems, I suggest we solve those problems and
        advance the state of the art in grammars and parsing.

        About six years ago I started working with the infrastructure
        for parsing programming languages in the program
        transformation system Stratego/XT. Stratego/XT employs the
        grammar formalism SDF and the scannerless generalized LR
        parser SGLR, which is the target of the SDF parser
        generator. The combination of SDF and SGLR is very easy to
        use. Indeed, I have been using the SGLR parser for years
        without knowing at all how it works!
%
        The essential advantages of scannerless parsing over parsing
        with a separate scanner are \emph{(1)} lexical disambiguation
        by context, \emph{(2)} full description of the syntax of a
        language in a uniform grammar formalism, and \emph{(3)}
        expressive lexical syntax. These advantages are easy to
        understand.
%
        They were originally motivated by application to minor issues
        in parsing existing programming languages, such as the lexical
        ambiguity between subrange types versus floating point
        literals in Pascal. Although these examples clearly illustrate
        the advantage of scannerless parsers, the problems with
        implementing parsers for these languages are not serious
        enough to make a strong case for scannerless parsing.

        The first time Eelco Visser presented his then latest ideas on
        the use of concrete object syntax~\cite{Vis02.gpce}, where the
        syntax of an object language is embedded in a metalanguage, I
        was rather skeptical about combining languages in this
        way. Slowly, however, I realized the full potential of
        scannerless generalized parsing for combining languages. This
        thesis is largely about explaining this potential,
        contributing technical bits and pieces here and there. Indeed,
        I consider my work to be an exploration in what we could
        achieve once we are liberated from the limitations of
        conventional parsing techniques. This explains the title,
        which is a reference to the beautiful song \emph{Exercises in
          Free Love} by Freddie Mercury\footnote{Exercises in Free
          Love is a song performed without lyrics as a predecessor to
          \emph{Ensue\~no} (Dream), from the album Barcelona by
          Montserrat Caball\'e and Freddie Mercury.} (various puns
        intended). The Dutch title is a pun as well, referring to the
        almost military operations I usually set up to finish our
        papers before a conference deadline.

        There is still a lot of work left to do. Reflecting on my
        work, I feel that one of the main contributions is that I
        provide a strong motivation for rethinking the way we work
        with syntax and languages, and renewing research into fully
        automatic parser generation.
%
        I wish I discovered earlier that this is what I wanted to
        attack (and for example had spent less time implementing a
        Java typechecker), but well, I suppose a thesis project is
        never finished.

        Finally, I would like to emphasize that this area ended up
        being the subject of my thesis \emph{despite} Eelco Visser's
        supervision. Since Eelco designed the current revision of SDF
        and integrated scannerless parsing and generalized LR parsing,
        this thesis almost looks like a natural continuation of his
        work. Surprisingly, Eelco continuously encouraged me to move
        beyond parsing and syntax, but despite his best efforts my
        work moved closer and closer to parsing issues.

\section*{Acknowledgements}

\subsubsection*{Supervisors}

	First of all I thank Eelco Visser, my \emph{`co-promotor'},
        for being my supervisor. He was not only my supervisor, but
        also a mentor and a friend. Almost everything I know about
        research, I learned from him.
%
        Our collaboration was absolutely brilliant. We always worked
        together on all my publications. The balance between
        supervising me and contributing to the work we did was
        perfect. I appreciate how Eelco always gives his students the
        opportunity to discover things on their own and express their
        own ideas, as opposed to just telling everything he knows
        immediately.

        I do not remember any formal meeting. We would just end up
        talking about work every time we wanted to. This must have been
        difficult sometimes, due to the constant context switches a
        supervisor has to go through. I am impressed by the capability
        of some researchers to come up with brilliant remarks just
        after such a context switch.

        Eelco Visser also taught me a lot about publishing
        strategies. I learned how to target a topic to a general
        audience, which is most important if you are in a certain
        niche and your motivation will not be immediately obvious to
        other researchers. I feel well-equipped to start working as an
        independent researcher, which is entirely due to Eelco's
        teaching. Also, Eelco's excellent network gave me the
        opportunity to meet many interesting researchers, learn how to
        review papers, and even resulted in finding my future
        wife. How can one possibly have more impact?

        Although I will be moving on to a different country and
        different universities, I hope we will continue to work
        together now and then.

        I also thank Doaitse Swierstra, my \emph{`promotor'}. Although
        we didn't meet frequently, I was always struck how well aware
        he was of our work and my situation. His advice was much
        appreciated. Also, I have learned that I should probably never
        buy a cottage for holidays.

\subsubsection*{Peers}

        I thank the members of the reading committee Mark van den
        Brand, Dick Gr\"une, Johan Jeuring, Kees Koster, and Oege de
        Moor for reviewing my thesis.

	Many anonymous reviewers of conferences provided useful
        suggestions to improve my work. Also, several people
        voluntarily provided useful feedback on one or more of my
        papers: Eelco Dolstra, Jeff Gray, Shan Shan Huang, Merijn de
        Jonge, Karl Trygve Kalleberg, Emmanuel Onzon, and Martijn
        Vermaat.

        All the work of this thesis has been done in collaboration
        with other people. I thank the co-authors of my publications
        for their contributions: Eric Bouwers, Arthur van Dam, Eelco
        Dolstra, Ren\'e de Groot, Karl Trygve Kalleberg, Koen
        Muilwijk, Karina Olmos, \'Eric Tanter, Rob Vermaas, Jurgen
        Vinju, and Eelco Visser.

\subsubsection*{Friends}

        Arthur van Dam was always happy to help me with \LaTeX{} and
        thesis design issues. There must be some contribution from him
        at almost every single page. He also suggested the Dutch
        translation \emph{exercities} for \emph{exercises}, which is
        so much more appropriate than the original \emph{oefeningen}!
        I will miss speed skating together and even more will I miss
        our road bicycle racing adventures across the country. How am
        I supposed to deal with the wind without you? I am sorry I
        beat you at the only serious hill we climbed together (he
        still claims he just let me win). I should make sure you get
        your sweet revenge.

        Eelco Dolstra provided me the source of his excellent thesis,
        which saved me a lot of time during the final preparations of
        my thesis. He was also a very pleasant officemate. I miss his
        disquisitions and his exclamations of delight when he has
        found more remarkable \emph{trivia} at Wikipedia. Eelco also
        carefully reviewed the Dutch abstract of this thesis.

        My friend and ex-colleague Rob Vermaas was always annoying and
        funny in his own special way. TraCE, a grammar for Shell, and
        Unicode support are only a few of the things he would happily
        bother me with almost every week. I will miss the excellent
        pies prepared occasionally by Lizi Vermaas at the occasion of
        an accepted paper!

        This thesis would not be complete without mentioning the IRC 
        channel whose name must not be mentioned. One of the members,
        Armijn Hemel, deserves special mentioning for his unique
        approach towards friendship, but I have no clue what to say
        about him here. Let me just slap him.

\subsubsection*{Logistics}

	I pulled quite a few all-nighters (also known as a
        \emph{bravo}) and I do not think I could have done this
        without listening to the music of Blank\ \&\ Jones, the Pet
        Shop Boys, Moby, and Freddie Mercury. I highly recommend
        everyone straightforward, gay dance music to get through a
        long night. I also recommend tea, not coffee. Yunnan green tea
        works for me.
        
        My loyal computer Logistico could not deal with the
        finalization of my thesis and the prospect of being
        abandoned. She died just after submitting my thesis to the
        reading committee. I owe her every single byte I produced for
        this thesis.

\subsubsection*{Family}
        
        I thank my parents, my sister, and my brother-in-law for their
        continuous support. Finally, there is Shan Shan, my lovely
        fianc\'ee. I do not know what to say here about this
        miracle. Anything I would say about you would be
        insufficient. I am looking forward to spending my life with
        you.

\medskip
        
\begin{flushright}
  Martin Bravenboer\\
  December 6, 2007\\
  Delft
\end{flushright}
