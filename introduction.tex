  \singlelinetitle\chapter{Introduction}
  \label{introduction}
Deformation is a point of focus in computer animation: surface deformation of soft body animation, and skin deformation of character animation. To date, a series of deformation techniques have been proposed, mostly,  targeting the practical usage and or the fidelity. Skinning and space deformation are two extensively applied techniques of those, largely improving the realism of animation. 

Skinning, often referred to as skeletal skinning, is to bind the character skin, often represented by a polygonal mesh, to the rig, likewise referred to as a hierarchical set of interconnected rigid bones. The most important phrase of binding is to decide the influence weights, which indicate how a vertex is influenced by a bone transformation. Deformations are produced by transforming the bones. This process can be performed manually by artists, or automated by a motion capture pipeline. Through skinning, deformations are stored as data of weights and transformations, rather than the vertices themselves, largely compressing the data to be stored.

The simplest and fastest skinning scheme is the linear blend skinning (LBS), which linearly blends transformation matrices with per-bone weights. LBS still dominates the practical usage, in particular in computer games, though it exhibits the well known \textit{candy-wrapper} artifacts, in the case of large rotations at joints. However, due to the increasing demand of realistic skin deformation, such LBS artifacts should be eliminated. One branch of solutions is to expand the expressive power of LBS by provide additional weights per bone~\cite{Wang:2002:MEL,Merry:2006:AST}. A set of example poses, required by their least-squares functions, has to be provided. The exemplars, however, often do not exit, more or less hindering their applications. Another branch is to use rotation-based blending, leading to intrinsic averages of rotations in 3D. Such blending methods come at the price of nonlinear parameterization of rotation, for example, as unit quaternion~\cite{Kavan:2005:SBS}, or dual quaternion\cite{Kavan:2008:GSA}, thus the computational cost is increased. In particular, dual quaternion effectively handles the translational component of transformation, leading to an idea alternative to LBS, called dual quaternion skinning (DQS). DQS can be linearized~\cite{Kavan:2009:ALN} to reduce the runtime cost. However, to create highly expressive posing is inherently limited using the above, traditional skinning algorithms, as they are not designed to extremely large rotations, such as large bends and twists. To accommodate the interesting, expressive character poses, a method called \textit{differential blending}~\cite{Oztireli:2013:DBE} was proposed based on differential transformations so that the intrinsic averaging is applied to small rotations, overcoming the discontinuities that arise in complex regions, where vertices are influenced multiple bones.

Modern research of blend skinning seeks an automatic rigging pipeline. It includes skeletonisation from a static shape~\cite{Baran:2007:ARA,Shapira:2008:CMP,Jacobson:2014:TMI}, automatic weights~\cite{Baran:2007:ARA,Jacobson:2011:BBW,Dionne:2013:GVB}, and automatic specification of the degrees of freedom~\cite{Gleicher:1999:AOM,Shi:2007:MPC,Jacobson:2012:FAS}. Desired results specific to humanoid shapes have been obtained, but intractable problems still remain in achieving fully automatic pipelines~\cite{skinningcourse:2014}. Alternative automatic method is based on \textit{skinning decomposition}, an algorithm of extracting the linear blend skinning from a set of mesh sequences~\cite{James:2005:SMA,kavan:2010:SAM}.  By imposing the orthogonal constraints on the bone rotation matrices~\cite{Le:2012:SSD}, decomposition is compatible with rigid bones, making the approximated LBS model suitable for motion editing and skeleton extraction. Hence, the resulting LBS model is essentially compatible with the traditional rigging pipeline. Then, this method was improved~\cite{Le:2014:RAS}, by removing redundant bones, and smoothing the weights using regularization, thus, it is robust to extract the skeleton and joint positions.

The use of rigid bones in blend skinning limits the space of possible deformations. Rigid bones are essentially effective for bending of limbs, as the bone weights do not vary much over the region of a body segment. But for stretching, twisting, and the manipulations of supple regions, many short bones are needed, increasing the skinning complexity. A lot of possibilities exist to replace the rigid bone with other types of user handle.  Curve skeleton~\cite{Yang:2006:CSS,Forstmann:2007:DSS} supports a wide range of poses including expressive animations, but it requires special rig controls that are inconsistent with the existing rigging pipeline. The \textit{line-of-action}~\cite{Guay:2013:LAI,Oztireli:2013:DBE} concept was implemented, along with a sketch-based interface, to support the rigging of curve skeleton. Cage-based deformation provides direct and precise manipulations, thus, it is well suited for bulging and thinning. However, the popular \textit{mean value coordinates}~\cite{Ju:2005:MVC} may contain negative values, leading to inevitable artifacts. Later, \textit{Harmonic coordinates}~\cite{Joshi:2007:HCC} was proposed to use instead, as they are guaranteed to be positive.  Nevertheless, they do not have a closed-form expression, thus, to compute them is difficult. \textit{Green coordinates}~\cite{Lipman:2008:GC}, which consider normals in addition to vertices, have closed-form expression, and support detail-preserving deformations. A main problematic issue in practice is to establish and then manipulate a proper cage, enclosing the target mesh, as these processes are tedious.

Recent trend is to use point handles, and several works have demonstrate the effectiveness. One of the outstanding merits lies in that points are perhaps the easiest handles to place, and the positioning of each is independent. In contrast, transforming a joint would shift its incident bones outside the volume, as the skeleton is a hierarchical structure. Influence weights computed for points have the characteristics well suited to handle the tasks, such as twisting, stretching, and supple deformation, which are difficult to achieve by blend skinning with bones. Notably, it is that point weights vary much over the surface. Good point weights can be computed using the \textit{Bounded Biharmonic Weights} method~\cite{Jacobson:2011:BBW}, satisfying the critical properties: locality, sparsity, and smoothness. Exiting works expanded the space of deformations possible with LBS via using a hybrid rig controls composed of points and other types of handles (e.g. rigid bones and cages)~\cite{Jacobson:2011:BBW,Jacobson:2011:STB,Kavan:2012:EDC}. Until now, achieving a fully point-based skinning scheme is expected for further research. Targeting automatic skinning, how to select good placements of points in 3D also needs to be figured out. Starting based on the \textit{Schelling points}~\cite{Chen:2012:SPS} is perhaps the initial solution.

Direct skinning supports no shape-details preserving deformations. Therefore, distortion would appear, in particular at joints. Such artifacts are easy to perceived in textured models with rich details. An extension of skinning~\cite{Kavan:2012:EDC}, which mimics the nonlinear elastic deformation by optimizing the influence weights, has been developed, delivering visually similar result. The advantage is to let  professional rigging artists enjoy the convenience of LBS or DQS and the joint-bulging artifacts are suppressed behind.

Instead of an approximation, methods that cast the deformations as an elastic energy minimization problem are employed to produce higher quality deformations. Without a loss of generality, they can be referred to as the methods that construct a rigidity energy over the shape, and the energy is a form of  quadratic variational minimization problem~\cite{Botsch:2008:LVS}. Though surface deformation problem is inherently non-linear, the linear variational deformation method, which provides an approximate result, is more attractive, as it is robust and computationally efficient. If the artifacts arising from large rotations should be eliminated, deducing local rotations of the surface has to be considered, forming a nonlinear rigidity energy. This would of course largely increases the computational cost. The local ration of  vertex is computed by singular value decomposition of the matrix built based on the one-ring neighbors of that vertex. To incorporate user edits, the energy is subjected to positional constraints for intuitive manipulations. The definition of the rigidity energy based on Laplacian representation (e.g.~\cite{Sorkine:2007:ASM}) is conceptually easy to understand and easy to implement. Surface-based energy minimization is bound to the surface representation, thus, it may encounter geometric issues.

The above mentioned variational deformation energy, nonetheless, needs to define global or local, volume constraints~\cite{Huang:2006:SGD,Ben-Chen:2009:VHM} in addition to the positional constraints, in order to preserve the volume throughout the deformation. Such constraints are often complex to design, and are not as-intuitive-as positional constraints to edit on-the-fly. Hence, in general, the shape-details preserving deformations often come at the cost of comprising volume preservation. Furthermore, some surface-based variational deformation with volume constraints is not able preserve the shape locality and the volume simultaneously. For example, Lipman et al.~\cite{Lipman:2007:VSP} first guaranteed that the shape details are preserved during deformation by ignoring a loss of volume, then scaled the moving frames, that represent the shape, according to local curvature information to restore the volumetric properties.

Surface-based energy might not be as natural as a volumetric discretization with a physically plausible deformation energy. With a discretization of the volume domain, the objective function of the volumetric energy is formulated to enforce the local rigidities of volumetric elements. A great attractive point of using volumetric representation is that the minimization is free from sophisticatedly defined volume constraints. The vulnerability is that a volumetric discretization is required, usually resulting in dense data to be processed. Also, a mapping function between the volumetric deformation and the surface deformation needs to be defined using the concept of embedding for instance. To maintain the high-order smoothness, a mapping that are computationally expensive would be used. For example, Botsch et al.~\cite{Botsch:2007:adaptive}  used the tri-harmonic radial basis functions, which provide high-quality $C^{2}$-continuous deformation fields, to defined the mapping as scattered data interpolation.

Since most of the real-world object are solids, hence volume preservation is highly desired. The degrees of volume achieved by a deformation energy depend on what volume constraints were modeled. Simply imposing constraints on edge lengths would result in approximated preservation, but it speeds up the calculations and simplifies the implementation. To keep the volume constant throughout the deformation, better solutions are perhaps using a geometric volume correction. It is to apply displacement constraints on vertices. The displacement corrections could be computed automatically~\cite{von:2008:volume} or controlled manually via one-dimensional profile curves~\cite{Rohmer:2009:EVP}. These methods requires more user interactions with respect to deformation energies with proper volume constraints, especially the latter method asks users to specify the influence weights, used to control the degrees of volume preservation when dragging the curves.







 